\chapter{Image Data}

\section{Image Set}
The images we had to classify consisted of images representing the 26 upper case
characters A to Z. For each character there was a set of 20 images with minute
differences in appearance.

De images is in the form of grayscale images which uses the JPEG compression
method for storage and file format. Which means that they have a specific
compression algorithm, and file format.

Even though all images are the size 30x30 pixels, they will all have different
sizes because of the compression algorithm.  This means that we have to read
images using an image library for C++ or pre-convert the images into a specific
file format.


\section{Conversion}
To read the images in the software(C++) we tried different libraries which had
image reading capabilities. Reading them as plain binary files would'nt have
worked since they have embeded metadata and are compressed with the JPEG format.

Because of this we tried ImageMagick and OpenCV, but neither worked with any
ease so we decided to use Python to convert them into a play text file with the
images pixel values as text data.

So by creating the script to convert images, see appendix
\ref{code:imgConverter}, we were left with a single file holding one informative
line with which character was depicted in the set and the output target values.
This is then followed by 20 lines, each holding 100 decimal values depicting the
pixel values of the image.

The script will open the pictures in ascending order starting with the images
dipicting As, then Bs and so on, until Z. And for each of these characters it
will read its 20 images straight away.

Once it has loaded the image, using Python Image Library(PIL), it will go
through a 10x10 array which will create the new picture. Then it will pick out a
3x3 subset of the image, see figure \ref{fig:imgSubset}, and average its value.
Then it will add it to a list for printing. It also adds it to a new 30x30 image
that depicts the scaled doen version for comparison, though it is still in
30x30px and the image data is only 10x10 scale..



\subsection{Data Verification}
To verify that the converted images was in fact correctly created, we took two
approaches. First as well as grabbing the average value of the 3x3 subsets we
created a scaled down picture set of the converted data. This served as a visual
verification of our data conversion.  The second verification method was to
create a small Python script, see listing \ref{code:imgDisp} , that read the
data from our converted file and displayed the images as a 10x10 array of
symbols in the console.

\section{In software}
Since we decided to pre-convert the images into data files as described above we
needed a custom datastructure to hold the data in memory while running the
program.  We created the following class which creates a long list of ''Data''
objects where each objects contains the data for one image.

In the program there are two global variable directly linked to the Data
structure. The global variable ''curTarget'' and ''inData'' which holds the
current array of target values, and the current Data object which holds the
needed image data.

\subsection{Data structure}
We've depicted the data structure in the UML figure \ref{fig:uml_data}.
\begin{description}
\item[fasit:]		This holds which character the image data represents, in upper
	case ASCII letters between A and Z.

\item[target:]	This is a pointer to an integer array which will hold the target
	values.  This is used when pushing it through the network to decide hwo much
	each output node is off target. The array size is the same as the number of
	output \gls{perceptron}s.

\item[values:]	This is a pointer to a float array, which holds all of the input
	values for the input layer.  This is used to set the initial values of the
	input layer which then is transformed with an activation function.  The size
	of the array is the number of input \gls{perceptron}s.

\item[next:]		This is a pointer to the next item in the list of Data objects.
	It is used for grabbing the next object in the list to perform some actions
	on.
\end{description}


